\definemode[grande][yes]
\definemode[mediano][no]
\definemode[pequeno][no]

\startluacode
FIRSTK=2 -- First kanji. Start in 2, the row 1 is a header
LASTK=2137    -- Last janji (2137 for jouyou kanji)
\stopluacode

\def\lua#1{\ctxlua{context(#1)}}

\startmode[grande]
\startluacode gridx=4 gridy=4 \stopluacode
\definepapersize[A8bis][width=52.5mm,height=74.25mm]
\setuppapersize[A8bis][A4]
\setuplayout[location=middle,margindistance=0mm,margin=3mm,width=46.5mm,height=71mm,topspace=2mm,backspace=3mm]
\stopmode

\startmode[mediano]
\startluacode gridx=6 gridy=11 \stopluacode
\definepapersize[tarjetamediana][width=35mm,height=27mm]
\setuppapersize[tarjetamediana][A4]
\setuplayout[location=middle,margindistance=0mm,margin=2mm,width=26.75mm,height=24mm,topspace=1.5mm,backspace=4mm]
\stopmode

\startmode[pequeno]
\startluacode gridx=6 gridy=24 \stopluacode
\definepapersize[tarjetapequena][width=35mm,height=12.375mm]
\setuppapersize[tarjetapequena][A4]
\setuplayout[location=middle,margindistance=0mm,margin=2mm,width=26.75mm,height=10mm,topspace=1mm,backspace=4mm]
\stopmode

\setuppaper[nx=\ctxlua{context(gridx)},ny=\ctxlua{context(gridy)}]
\setuparranging[XY]
%\showframe

% to use the IPAex font, run as root:
% apt-get install fonts-ipaexfont-mincho
% export OSFONTDIR=/usr/share/font
% mtxrun --script font --reload

% fonts for big cards
\definefont[fontkanji][name:ipaexmincho at 50pt]
\definefont[fontkey][name:ipaexmincho at 15pt]
\definefont[fontpalabras][name:ipaexmincho at 13pt]
\definefont[fontpalabrasmuchas][name:ipaexmincho at 10pt]
\definefont[fontheader][name:ipaexmincho at 8pt]
\definefont[fontreadings][name:ipaexmincho at 13pt]
\definefont[fontreadingsmuchas][name:ipaexmincho at 10pt]
\definefont[fontwordreading][name:ipaexmincho at 10pt]
\definefont[fonttranslation][name:latinmodernromanregular at 8pt]
\definefont[fontfooter][name:latinmodernromanregular at 7pt]
\definefont[fontlabel][name:latinmodernromanregular at 15pt]
\definefont[fontmnemo][name:latinmodernromanregular at 8pt]

% fonts for medium cards
\definefont[fontkanjimed][name:ipaexmincho at 36pt]
\definefont[fontkeymed][name:ipaexmincho at 10pt]
\definefont[fontreadingsmed][name:ipaexmincho at 10pt]
\definefont[fontlabelmed][name:latinmodernromanregular at 11pt]
\definefont[fontlabelmuymed][name:latinmodernromanregular at 8pt]

% fonts for small cards
\definefont[fontkanjipeq][name:ipaexmincho at 23pt]
\definefont[fontlabelpeq][name:latinmodernromanregular at 11pt]
\definefont[fontlabelmuypeq][name:latinmodernromanregular at 8pt]

% paragraphs
\defineparagraphs[palabras][n=2]
\setupparagraphs[palabras][1][width=.45\textwidth, color=darkred, align=middle]
\setupparagraphs[palabras][2][width=.4\textwidth, color=darkgreen, align=middle]

\defineparagraphs[lecturas][n=2]
\setupparagraphs[lecturas][1][width=.3\textwidth, color=darkred, align=flushleft]
\setupparagraphs[lecturas][2][width=.6\textwidth, color=darkgreen, align=flushright]

% 4 gray points in the corners
\newdimen\puntosize
\newdimen\lenx
\newdimen\leny
\puntosize=0.5mm
\lenx=\paperwidth
\leny=\paperheight
\advance\lenx by -\puntosize
\advance\leny by -\puntosize
\def\punto{\blackrule[width=\puntosize, height=\puntosize, color=middlegray]}
\definelayer[puntos][x=0mm, y=0mm, width=\paperwidth, height=\paperheight, repeat=yes]
\setlayer[puntos][x=0mm,   y=0mm  ]{\punto}
\setlayer[puntos][x=0mm,   y=\leny]{\punto}
\setlayer[puntos][x=\lenx, y=0mm  ]{\punto}
\setlayer[puntos][x=\lenx, y=\leny]{\punto}
\setupbackgrounds[page][background=puntos]

\setupheader[state=none]
\setupfooter[state=none]
\starttext
\language[es]
\startluacode

-- Must be a csv file with 28 fields separated by commas
-- and every field enclosed by double quotes.
-- The characters " and $ are forbiden inside each field
NFIELDS=28
local filename='kanjinancho.txt'
local file=io.open(filename)
local content=file:read "*a"
file:close()

i=1
tokens={}
for word in string.gmatch(content, '".-"') do
	tokens[i]=string.sub(word,2,-2)
	i=i+1
end
kanjis_por_hoja=gridx*gridy
numero_de_kanjis=LASTK-FIRSTK+1
kanjis_sobrantes=numero_de_kanjis % kanjis_por_hoja
if kanjis_sobrantes > 0 then
		numero_de_kanjis=numero_de_kanjis + kanjis_por_hoja - kanjis_sobrantes
end
n=0
-- These variables are used below, I am unable to make lua escape them
-- correctly if I define them in the secon luacode block
patron="([^\n]+)\n([^\n]+)"
replace="%1/%2"
\stopluacode

\dorecurse{\ctxlua{context(numero_de_kanjis*2)}}{

\startluacode
	hoja=math.floor(n/(2*kanjis_por_hoja))
	k=n-hoja*2*kanjis_por_hoja
	if k>= kanjis_por_hoja then
		dorso=1
		k=k-kanjis_por_hoja
		cociente=math.floor(k/gridx)
		resto=k-cociente*gridx
		k=cociente*gridx+gridx-resto-1
	else
		dorso=0
	end
	i=k+hoja*kanjis_por_hoja
	b=(i+FIRSTK-1)*NFIELDS
	kanji=				tokens[b +  1]
	spanish_key=		tokens[b +  2]
	mnemo=				tokens[b +  3]
	key=				tokens[b +  4]
	key_reading=		tokens[b +  5]
	heisig6=			tokens[b +  6]
	strokes=			tokens[b +  7]
	frequency=			tokens[b +  8]
	grade=				tokens[b +  9]
	on_jouyou=			tokens[b + 10]
	kun_jouyou=			tokens[b + 11]
	skip=				tokens[b + 12] skip=string.gsub(skip,"-0", "-")
	on=					tokens[b + 13]
	on_word=			tokens[b + 14] on_word=string.gsub(on_word,"\n","\\par ")
	on_word_reading=	tokens[b + 15]
	on_translation=		tokens[b + 16]
	kun=				tokens[b + 17]
	kun_word=			tokens[b + 18] kun_word=string.gsub(kun_word,"\n","\\par ")
	kun_word_reading=	tokens[b + 19]
	kun_translation=	tokens[b + 20]
	kana_origin=		tokens[b + 21]
	radical=			tokens[b + 22]
	english_meanings=	tokens[b + 23]
	irregular_readings=	tokens[b + 24]
	examples=			tokens[b + 25]
	on_kanjidic=		tokens[b + 26]
	kun_kanjidic=		tokens[b + 27]
	nanori=				tokens[b + 28]

	% -- put each on and kun word and its translation in tables
	% -- to be able to access them individually
	lista_on={}
	for s in on_word_reading:gmatch("[^\n]+") do
		table.insert(lista_on,s)
	end
	lista_on_translation={}
	for s in on_translation:gmatch("[^\n]+") do
		table.insert(lista_on_translation,s)
	end
	lista_kun={}
	for s in kun_word_reading:gmatch("[^\n]+") do
		table.insert(lista_kun,s)
	end
	lista_kun_translation={}
	for s in kun_translation:gmatch("[^\n]+") do
		table.insert(lista_kun_translation,s)
	end

	% -- if there are too many on or kun readings, I group them in pairs
	aux, onpairs=on:gsub(patron, replace)
	if onpairs>1 then on=aux end
	on=string.gsub(on,"\n","\\par ")
	
	aux, kunpairs=kun:gsub(patron, replace)
	if kunpairs>1 then kun=aux end
	kun=string.gsub(kun,"\n","\\par ")

	maxpalabras=math.max(onpairs,kunpairs)
	n=n+1
	\stopluacode
	\page[yes]

	\startmode[grande]
	\ifnum\lua{dorso}=0
		%%%% CARA %%%%
		\fontheader
		\startoverlay
		{\leftaligned{\lua{radical}}}
		{\blue \midaligned{\lower 4mm\hbox{\fontkey\lua{key}}}}
		{\rightaligned{\lua{heisig6}}}
		\stopoverlay
		\vskip 4mm minus 6mm
		\midaligned{\fontkanji\lua{kanji}}
		\vskip 4mm minus 6mm
		\fontpalabras
		\ifnum\lua{maxpalabras}>2 \fontpalabrasmuchas \fi
		\startpalabras
			\lua{on_word} \ % hack: if there are not on words, put a space
		\palabras
			\lua{kun_word}
		\stoppalabras
		\vfill
		\fontfooter
		\startoverlay
		{\leftaligned{F:\lua{frequency}}}
		{\midaligned{T:\lua{strokes} \ (\lua{skip})}}
		{\rightaligned{G:\lua{grade}}}
		\stopoverlay
	\else
		%%%% DORSO %%%%
		\fontreadings
		\ifnum\lua{maxpalabras}>2 \fontreadingsmuchas \fi
		\startlecturas
			\lua{on} \ % hack
		\lecturas
			\lua{kun}
		\stoplecturas
		\vskip -1mm
		\midaligned{\fontkey\blue\lua{key_reading}}
		\vskip 2mm
		\midaligned{\fontlabel\lua{spanish_key}}
		\vskip 2mm
		\setupbodyfont[6pt]
		\fontmnemo
		\hrule
		\vskip 1mm \lua{mnemo} \vskip 1mm
		\hrule
		\vskip 2mm minus 2mm
		\vfil
		\startluacode
		for count=1, \string##lista_on do
			tex.print("\\setbox0\\hbox{\\fontwordreading ", lista_on[count], " }")
			tex.print("\\setuptab[width=.9\\wd0]")
			tex.print("\\tab{\\fontwordreading\\darkred ", lista_on[count],"}")
			tex.print("\\fonttranslation \\black ", lista_on_translation[count])
			tex.print("\\par \\vfil")
		end
		\stopluacode
		\vfil
		\startluacode
		for count=1, \string##lista_kun do
			tex.print("\\fontwordreading\\darkgreen", lista_kun[count])
			tex.print("\\fonttranslation \\black \\hskip 1mm ", lista_kun_translation[count])
			tex.print("\\par \\vfil")
		end
		\stopluacode
		\vfil
	\fi
	\stopmode

	\startmode[mediano]
	\ifnum\lua{dorso}=0
		%%%% CARA %%%%
		\fontkeymed
		\topglue 2mm
		\midaligned{\blue\lua{key}}
		\vskip 2mm
		\fontkanjimed
		\midaligned{\blue\lua{kanji}}
	\else
		%%%% DORSO %%%%
		\fontkeymed
		\topglue 0.1mm
		\vfil
		\midaligned{\blue\lua{key_reading}}
		\fontlabelmed
		\setbox0\hbox{\lua{spanish_key}}
		\ifdim\wd0>\textwidth \fontlabelmuymed \fi
		\midaligned{\black\lua{spanish_key}}
		\fontreadingsmed
		\vskip -3mm
		\startlecturas
			\lua{on} \ % hack
		\lecturas
			\lua{kun}
		\stoplecturas
		\vfil
	\fi
	\stopmode

	\startmode[pequeno]
	\ifnum\lua{dorso}=0
		%%%% CARA %%%%
		\topglue 2mm
		\vfil
		\hfil\fontkanjipeq\blue\lua{kanji}\hfil
		\vfil
	\else
		%%%% DORSO %%%%
		\fontlabelpeq
		\topglue 2mm
		\vfil
		\setbox0\hbox{\lua{spanish_key}}
		\ifdim\wd0>\textwidth \fontlabelmuypeq \fi
		\midaligned{\black\lua{spanish_key}}
		\vfil
	\fi
	\stopmode
}

\stoptext

